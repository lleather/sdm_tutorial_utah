\documentclass[]{article}
\usepackage{lmodern}
\usepackage{amssymb,amsmath}
\usepackage{ifxetex,ifluatex}
\usepackage{fixltx2e} % provides \textsubscript
\ifnum 0\ifxetex 1\fi\ifluatex 1\fi=0 % if pdftex
  \usepackage[T1]{fontenc}
  \usepackage[utf8]{inputenc}
\else % if luatex or xelatex
  \ifxetex
    \usepackage{mathspec}
  \else
    \usepackage{fontspec}
  \fi
  \defaultfontfeatures{Ligatures=TeX,Scale=MatchLowercase}
\fi
% use upquote if available, for straight quotes in verbatim environments
\IfFileExists{upquote.sty}{\usepackage{upquote}}{}
% use microtype if available
\IfFileExists{microtype.sty}{%
\usepackage{microtype}
\UseMicrotypeSet[protrusion]{basicmath} % disable protrusion for tt fonts
}{}
\usepackage[margin=1in]{geometry}
\usepackage{hyperref}
\hypersetup{unicode=true,
            pdftitle={Manipulating and Using Spatial Data in R},
            pdfauthor={Lila Leatherman},
            pdfborder={0 0 0},
            breaklinks=true}
\urlstyle{same}  % don't use monospace font for urls
\usepackage{color}
\usepackage{fancyvrb}
\newcommand{\VerbBar}{|}
\newcommand{\VERB}{\Verb[commandchars=\\\{\}]}
\DefineVerbatimEnvironment{Highlighting}{Verbatim}{commandchars=\\\{\}}
% Add ',fontsize=\small' for more characters per line
\usepackage{framed}
\definecolor{shadecolor}{RGB}{248,248,248}
\newenvironment{Shaded}{\begin{snugshade}}{\end{snugshade}}
\newcommand{\KeywordTok}[1]{\textcolor[rgb]{0.13,0.29,0.53}{\textbf{#1}}}
\newcommand{\DataTypeTok}[1]{\textcolor[rgb]{0.13,0.29,0.53}{#1}}
\newcommand{\DecValTok}[1]{\textcolor[rgb]{0.00,0.00,0.81}{#1}}
\newcommand{\BaseNTok}[1]{\textcolor[rgb]{0.00,0.00,0.81}{#1}}
\newcommand{\FloatTok}[1]{\textcolor[rgb]{0.00,0.00,0.81}{#1}}
\newcommand{\ConstantTok}[1]{\textcolor[rgb]{0.00,0.00,0.00}{#1}}
\newcommand{\CharTok}[1]{\textcolor[rgb]{0.31,0.60,0.02}{#1}}
\newcommand{\SpecialCharTok}[1]{\textcolor[rgb]{0.00,0.00,0.00}{#1}}
\newcommand{\StringTok}[1]{\textcolor[rgb]{0.31,0.60,0.02}{#1}}
\newcommand{\VerbatimStringTok}[1]{\textcolor[rgb]{0.31,0.60,0.02}{#1}}
\newcommand{\SpecialStringTok}[1]{\textcolor[rgb]{0.31,0.60,0.02}{#1}}
\newcommand{\ImportTok}[1]{#1}
\newcommand{\CommentTok}[1]{\textcolor[rgb]{0.56,0.35,0.01}{\textit{#1}}}
\newcommand{\DocumentationTok}[1]{\textcolor[rgb]{0.56,0.35,0.01}{\textbf{\textit{#1}}}}
\newcommand{\AnnotationTok}[1]{\textcolor[rgb]{0.56,0.35,0.01}{\textbf{\textit{#1}}}}
\newcommand{\CommentVarTok}[1]{\textcolor[rgb]{0.56,0.35,0.01}{\textbf{\textit{#1}}}}
\newcommand{\OtherTok}[1]{\textcolor[rgb]{0.56,0.35,0.01}{#1}}
\newcommand{\FunctionTok}[1]{\textcolor[rgb]{0.00,0.00,0.00}{#1}}
\newcommand{\VariableTok}[1]{\textcolor[rgb]{0.00,0.00,0.00}{#1}}
\newcommand{\ControlFlowTok}[1]{\textcolor[rgb]{0.13,0.29,0.53}{\textbf{#1}}}
\newcommand{\OperatorTok}[1]{\textcolor[rgb]{0.81,0.36,0.00}{\textbf{#1}}}
\newcommand{\BuiltInTok}[1]{#1}
\newcommand{\ExtensionTok}[1]{#1}
\newcommand{\PreprocessorTok}[1]{\textcolor[rgb]{0.56,0.35,0.01}{\textit{#1}}}
\newcommand{\AttributeTok}[1]{\textcolor[rgb]{0.77,0.63,0.00}{#1}}
\newcommand{\RegionMarkerTok}[1]{#1}
\newcommand{\InformationTok}[1]{\textcolor[rgb]{0.56,0.35,0.01}{\textbf{\textit{#1}}}}
\newcommand{\WarningTok}[1]{\textcolor[rgb]{0.56,0.35,0.01}{\textbf{\textit{#1}}}}
\newcommand{\AlertTok}[1]{\textcolor[rgb]{0.94,0.16,0.16}{#1}}
\newcommand{\ErrorTok}[1]{\textcolor[rgb]{0.64,0.00,0.00}{\textbf{#1}}}
\newcommand{\NormalTok}[1]{#1}
\usepackage{graphicx,grffile}
\makeatletter
\def\maxwidth{\ifdim\Gin@nat@width>\linewidth\linewidth\else\Gin@nat@width\fi}
\def\maxheight{\ifdim\Gin@nat@height>\textheight\textheight\else\Gin@nat@height\fi}
\makeatother
% Scale images if necessary, so that they will not overflow the page
% margins by default, and it is still possible to overwrite the defaults
% using explicit options in \includegraphics[width, height, ...]{}
\setkeys{Gin}{width=\maxwidth,height=\maxheight,keepaspectratio}
\IfFileExists{parskip.sty}{%
\usepackage{parskip}
}{% else
\setlength{\parindent}{0pt}
\setlength{\parskip}{6pt plus 2pt minus 1pt}
}
\setlength{\emergencystretch}{3em}  % prevent overfull lines
\providecommand{\tightlist}{%
  \setlength{\itemsep}{0pt}\setlength{\parskip}{0pt}}
\setcounter{secnumdepth}{0}
% Redefines (sub)paragraphs to behave more like sections
\ifx\paragraph\undefined\else
\let\oldparagraph\paragraph
\renewcommand{\paragraph}[1]{\oldparagraph{#1}\mbox{}}
\fi
\ifx\subparagraph\undefined\else
\let\oldsubparagraph\subparagraph
\renewcommand{\subparagraph}[1]{\oldsubparagraph{#1}\mbox{}}
\fi

%%% Use protect on footnotes to avoid problems with footnotes in titles
\let\rmarkdownfootnote\footnote%
\def\footnote{\protect\rmarkdownfootnote}

%%% Change title format to be more compact
\usepackage{titling}

% Create subtitle command for use in maketitle
\newcommand{\subtitle}[1]{
  \posttitle{
    \begin{center}\large#1\end{center}
    }
}

\setlength{\droptitle}{-2em}

  \title{Manipulating and Using Spatial Data in R}
    \pretitle{\vspace{\droptitle}\centering\huge}
  \posttitle{\par}
    \author{Lila Leatherman}
    \preauthor{\centering\large\emph}
  \postauthor{\par}
      \predate{\centering\large\emph}
  \postdate{\par}
    \date{04/17/2019}


\begin{document}
\maketitle

\subsection{Welcome!}\label{welcome}

This workshop will show you how to load, manipulate, visualize, and
analyze spatial data in R. Our end application will be species
distribution models, but you can use similar techniques for other kinds
of spatial analysis and modeling.

Our example today will be species distribution models for aspen (Populus
tremuloides) in Utah. See
\href{https://www.annualreviews.org/doi/abs/10.1146/annurev.ecolsys.110308.120159}{Elith
et al. 2009} for a broader overview on species distribution modeling.

\subsubsection{A few notes:}\label{a-few-notes}

\begin{itemize}
\item
  This workshop was developed in R Markdown. As a user of this workshop,
  this means that if you are working from a PDF version of this report,
  you are seeing the same content that is in the script. As a future
  user of R Markdown, this means that you can write code, make figures,
  and annotate your findings all within one document-- that you can then
  export for multiple uses. Please take the opportunity to explore both
  the PDF and .Rmd versions of this workshop!
\item
  This workshop uses the
  \href{https://github.com/jennybc/here_here}{here} package to create
  dynamic references to file paths. This means you never have to use
  setwd() again! By creating an R project in a folder of your choice,
  here() identifies the source or home directory for the project on your
  machine, so that you don't have to manually set the location of the
  files or scripts.
\item
  If you download or clone the entire github repository, you will have
  all the data you need to run this model. The repository also already
  includes prepared data, but this workflow takes you through all the
  steps you would use to load, prep, and use the data on your own.
\end{itemize}

\subsection{Learning Objectives}\label{learning-objectives}

Over the course of this workshop, you will learn how to:

\begin{itemize}
\tightlist
\item
  read and write point data
\item
  read and write polygon data
\item
  read and write raster data
\item
  combine different types of spatial data
\item
  crop and mask spatial data
\item
  set and change the projection of a dataset
\item
  manipulate data using dplyr
\item
  create simple visualizations using base R and ggplot2
\end{itemize}

Additionally, we will learn some foundational concepts behind species
distribution modeling.

\subsection{Set up}\label{set-up}

First, we need to install all of the packages we need for this project.

\begin{Shaded}
\begin{Highlighting}[]
\CommentTok{# LOAD LIBRARIES}
  \CommentTok{#install.packages("here)}
    \KeywordTok{library}\NormalTok{(here)}
\end{Highlighting}
\end{Shaded}

\begin{verbatim}
## here() starts at /Users/lilaleatherman/Documents/github/sdm_tutorial_utah
\end{verbatim}

\begin{Shaded}
\begin{Highlighting}[]
  \CommentTok{#install.packages("tidyverse")}
    \KeywordTok{library}\NormalTok{(tidyverse)}
\end{Highlighting}
\end{Shaded}

\begin{verbatim}
## -- Attaching packages -------------------------------------------------------------------------------------------------------------------- tidyverse 1.2.1 --
\end{verbatim}

\begin{verbatim}
## v ggplot2 3.1.0       v purrr   0.3.2  
## v tibble  2.1.1       v dplyr   0.8.0.1
## v tidyr   0.8.3       v stringr 1.4.0  
## v readr   1.1.1       v forcats 0.3.0
\end{verbatim}

\begin{verbatim}
## Warning: package 'tibble' was built under R version 3.5.2
\end{verbatim}

\begin{verbatim}
## Warning: package 'tidyr' was built under R version 3.5.2
\end{verbatim}

\begin{verbatim}
## Warning: package 'purrr' was built under R version 3.5.2
\end{verbatim}

\begin{verbatim}
## Warning: package 'dplyr' was built under R version 3.5.2
\end{verbatim}

\begin{verbatim}
## Warning: package 'stringr' was built under R version 3.5.2
\end{verbatim}

\begin{verbatim}
## -- Conflicts ----------------------------------------------------------------------------------------------------------------------- tidyverse_conflicts() --
## x dplyr::filter() masks stats::filter()
## x dplyr::lag()    masks stats::lag()
\end{verbatim}

\begin{Shaded}
\begin{Highlighting}[]
  \CommentTok{#install.packages("rgeos")  }
    \KeywordTok{library}\NormalTok{(rgeos)}
\end{Highlighting}
\end{Shaded}

\begin{verbatim}
## rgeos version: 0.4-1, (SVN revision 579)
##  GEOS runtime version: 3.6.1-CAPI-1.10.1 
##  Linking to sp version: 1.3-1 
##  Polygon checking: TRUE
\end{verbatim}

\begin{Shaded}
\begin{Highlighting}[]
  \CommentTok{#install.packages("rgbif")}
    \KeywordTok{library}\NormalTok{(rgbif)}
  \CommentTok{#install.packages("maps")}
    \KeywordTok{library}\NormalTok{(maps)}
\end{Highlighting}
\end{Shaded}

\begin{verbatim}
## 
## Attaching package: 'maps'
\end{verbatim}

\begin{verbatim}
## The following object is masked from 'package:purrr':
## 
##     map
\end{verbatim}

\begin{Shaded}
\begin{Highlighting}[]
  \CommentTok{#install.packages("maptools")}
    \KeywordTok{library}\NormalTok{(maptools)  }
\end{Highlighting}
\end{Shaded}

\begin{verbatim}
## Loading required package: sp
\end{verbatim}

\begin{verbatim}
## Checking rgeos availability: TRUE
\end{verbatim}

\begin{Shaded}
\begin{Highlighting}[]
  \CommentTok{#install.packages("raster")}
    \KeywordTok{library}\NormalTok{(raster)}
\end{Highlighting}
\end{Shaded}

\begin{verbatim}
## 
## Attaching package: 'raster'
\end{verbatim}

\begin{verbatim}
## The following object is masked from 'package:dplyr':
## 
##     select
\end{verbatim}

\begin{verbatim}
## The following object is masked from 'package:tidyr':
## 
##     extract
\end{verbatim}

\begin{Shaded}
\begin{Highlighting}[]
  \CommentTok{#install.packages("rgdal") }
    \KeywordTok{library}\NormalTok{(rgdal) }
\end{Highlighting}
\end{Shaded}

\begin{verbatim}
## rgdal: version: 1.3-6, (SVN revision 773)
##  Geospatial Data Abstraction Library extensions to R successfully loaded
##  Loaded GDAL runtime: GDAL 2.1.3, released 2017/20/01
##  Path to GDAL shared files: /Library/Frameworks/R.framework/Versions/3.5/Resources/library/rgdal/gdal
##  GDAL binary built with GEOS: FALSE 
##  Loaded PROJ.4 runtime: Rel. 4.9.3, 15 August 2016, [PJ_VERSION: 493]
##  Path to PROJ.4 shared files: /Library/Frameworks/R.framework/Versions/3.5/Resources/library/rgdal/proj
##  Linking to sp version: 1.3-1
\end{verbatim}

\begin{Shaded}
\begin{Highlighting}[]
  \CommentTok{#install.packages("biomod2")}
    \KeywordTok{library}\NormalTok{(biomod2)}
\end{Highlighting}
\end{Shaded}

\begin{verbatim}
## biomod2 3.3-18 loaded.
## 
## Type browseVignettes(package='biomod2') to access directly biomod2 vignettes.
\end{verbatim}

\begin{Shaded}
\begin{Highlighting}[]
  \CommentTok{#install.packages("ggplot2")}
    \KeywordTok{library}\NormalTok{(ggplot2)}
  
\CommentTok{#set chunk options for writing}
\NormalTok{knitr}\OperatorTok{::}\NormalTok{opts_chunk}\OperatorTok{$}\KeywordTok{set}\NormalTok{(}\DataTypeTok{message =} \OtherTok{FALSE}\NormalTok{, }\DataTypeTok{warning =} \OtherTok{FALSE}\NormalTok{)}

\CommentTok{#reassign functions that are masked}
\NormalTok{extract <-}\StringTok{ }\NormalTok{raster}\OperatorTok{::}\NormalTok{extract}
\NormalTok{select <-}\StringTok{ }\NormalTok{dplyr}\OperatorTok{::}\NormalTok{select}
\end{Highlighting}
\end{Shaded}

\subsection{Load and organize spatial
data}\label{load-and-organize-spatial-data}

First, we're going to load some background spatial data for our project.

First, we'll get a polygon representing the state of Utah. There are
several different packages that can help you acquire administrative
boundaries, but here we're using getData() from the raster package.

\begin{Shaded}
\begin{Highlighting}[]
\CommentTok{#a way to get state data}
\NormalTok{states_list <-}\StringTok{ }\KeywordTok{c}\NormalTok{(}\StringTok{'Utah'}\NormalTok{)}
\NormalTok{states_all <-}\StringTok{ }\KeywordTok{getData}\NormalTok{(}\StringTok{"GADM"}\NormalTok{,}\DataTypeTok{country=}\StringTok{"USA"}\NormalTok{,}\DataTypeTok{level=}\DecValTok{1}\NormalTok{, }\DataTypeTok{path =} \KeywordTok{here}\NormalTok{(}\StringTok{"data/UT"}\NormalTok{))}
\NormalTok{UT.shp <-}\StringTok{ }\NormalTok{states_all[states_all}\OperatorTok{$}\NormalTok{NAME_}\DecValTok{1} \OperatorTok\StringTok{ }\NormalTok{states_list,]}

\CommentTok{#inspect}
\CommentTok{#UT.shp}
\KeywordTok{plot}\NormalTok{(UT.shp)}
\end{Highlighting}
\end{Shaded}

\includegraphics{SDM_workshop_utah_files/figure-latex/unnamed-chunk-2-1.pdf}

This loads and plots a shapefile for Utah. If you downloaded the whole
directory, this file has already been saved and provided for you, but
you can save it again for practice.

\begin{Shaded}
\begin{Highlighting}[]
\CommentTok{#create directory for output}
\KeywordTok{dir.create}\NormalTok{(}\KeywordTok{here}\NormalTok{(}\StringTok{"data/UT"}\NormalTok{))}

\CommentTok{#export}
\KeywordTok{writeOGR}\NormalTok{(UT.shp, }\KeywordTok{here}\NormalTok{(}\StringTok{"data/UT"}\NormalTok{), }\DataTypeTok{layer =} \StringTok{"UT"}\NormalTok{, }\DataTypeTok{driver =} \StringTok{"ESRI Shapefile"}\NormalTok{, }\DataTypeTok{overwrite =} \OtherTok{TRUE}\NormalTok{)}

\CommentTok{#load back in }
\NormalTok{UT.shp <-}\StringTok{ }\KeywordTok{shapefile}\NormalTok{(}\KeywordTok{here}\NormalTok{(}\StringTok{"data/UT/UT.shp"}\NormalTok{))}
\end{Highlighting}
\end{Shaded}

Next, we're going to load in our climate data. These data are commonly
used for species distribution modeling and represent environmental
variables that combine different climatic variables into variables that
are more meaningful for species. These data were downloaded from
\url{http://worldclim.org/version2} at 30s resolution and cropped to
Utah. These data are in Raster stack format - a collection of raster
layers. (What is this analagous to in Arc?)

Here, I'm using the stack(), commented out below, can also be used to
read in a raster stack, a .grd file. Alternately, the command readRDS
can also be used to load the data that I have saved in .Rdata format,
which is specific to R.

\begin{Shaded}
\begin{Highlighting}[]
\NormalTok{envStack <-}\StringTok{ }\KeywordTok{stack}\NormalTok{(}\KeywordTok{here}\NormalTok{(}\StringTok{"data/climate/envStack_init.grd"}\NormalTok{))}
\CommentTok{#envStack <- readRDS(here("data/climate/envStack_init.RData"))}

\CommentTok{#inspect}
\KeywordTok{plot}\NormalTok{(envStack)}
\end{Highlighting}
\end{Shaded}

\includegraphics{SDM_workshop_utah_files/figure-latex/unnamed-chunk-4-1.pdf}

We have environmental data, but we need to crop them only to the spatial
extent that we're interested in: in this case, Utah.

\begin{Shaded}
\begin{Highlighting}[]
\CommentTok{#crop to Utah extent}
\NormalTok{envStack_UT <-}\StringTok{ }\KeywordTok{crop}\NormalTok{(envStack, UT.shp)}

\CommentTok{#inspect - this just crops to the spatial extent of the object, not the outline of the polygon}
\KeywordTok{plot}\NormalTok{(envStack_UT)}
\end{Highlighting}
\end{Shaded}

\includegraphics{SDM_workshop_utah_files/figure-latex/unnamed-chunk-5-1.pdf}

\begin{Shaded}
\begin{Highlighting}[]
\CommentTok{# crop to Utah boundaries}
\NormalTok{envStack_UT <-}\StringTok{ }\KeywordTok{mask}\NormalTok{(envStack, UT.shp)}
\KeywordTok{plot}\NormalTok{(envStack_UT)}
\end{Highlighting}
\end{Shaded}

\includegraphics{SDM_workshop_utah_files/figure-latex/unnamed-chunk-5-2.pdf}

\begin{Shaded}
\begin{Highlighting}[]
\CommentTok{#export}
\KeywordTok{writeRaster}\NormalTok{(envStack_UT, }\DataTypeTok{file =} \KeywordTok{here}\NormalTok{(}\StringTok{"data/climate/envStack_UT.grd"}\NormalTok{), }\DataTypeTok{options =} \StringTok{"INTERLEAVE=BAND"}\NormalTok{, }\DataTypeTok{overwrite=}\OtherTok{TRUE}\NormalTok{)}
\KeywordTok{saveRDS}\NormalTok{(envStack_UT, }\DataTypeTok{file =} \KeywordTok{here}\NormalTok{(}\StringTok{"data/climate/envStack_UT.RData"}\NormalTok{))}
\end{Highlighting}
\end{Shaded}

Looks good! We can also export our final raster stack. In this case, we
can write the file as a raster, or save the data as a .Rdata file. Be
careful which file format you save a raster stack in-- even though some
file formats can be used for both single and multiple layer raster data
(e.g., .tif), these formats do not preserve the names of the layers in a
raster stack.

\subsubsection{Load occurrence data - from
GBIF}\label{load-occurrence-data---from-gbif}

Next, we'll load occurrence data for our species of interest from a few
different sources. First, we download data from the Global Biodiversity
Information Facility (GBIF) for our species of interest.

\begin{Shaded}
\begin{Highlighting}[]
\CommentTok{#function also doesn't work if the gbif website is down}
\NormalTok{gbif.POTR <-}\StringTok{ }\KeywordTok{occ_search}\NormalTok{(}\DataTypeTok{scientificName =} \StringTok{"Populus tremuloides"}\NormalTok{, }
                           \DataTypeTok{return =} \StringTok{"data"}\NormalTok{, }
                           \DataTypeTok{hasCoordinate =} \OtherTok{TRUE}\NormalTok{, }
                           \DataTypeTok{hasGeospatialIssue =} \OtherTok{FALSE}\NormalTok{, }
                           \DataTypeTok{limit =} \DecValTok{200000}\NormalTok{, }
                           \DataTypeTok{country =} \StringTok{"US"}\NormalTok{, }\DataTypeTok{stateProvince =} \KeywordTok{c}\NormalTok{(}\StringTok{"Utah"}\NormalTok{), }
                           \DataTypeTok{fields =} \KeywordTok{c}\NormalTok{(}\StringTok{"name"}\NormalTok{, }\StringTok{"decimalLongitude"}\NormalTok{, }\StringTok{"decimalLatitude"}\NormalTok{))}

\KeywordTok{colnames}\NormalTok{(gbif.POTR) <-}\StringTok{ }\KeywordTok{c}\NormalTok{(}\StringTok{"name"}\NormalTok{, }\StringTok{"lon"}\NormalTok{, }\StringTok{"lat"}\NormalTok{)}

\CommentTok{#export}
\KeywordTok{dir.create}\NormalTok{(}\DataTypeTok{path =} \KeywordTok{here}\NormalTok{(}\StringTok{"/data/occurrence/GBIF"}\NormalTok{))}
\KeywordTok{write.csv}\NormalTok{(gbif.POTR, }\StringTok{"./data/occurrence/GBIF/gbif.POTR.csv"}\NormalTok{, }\DataTypeTok{row.names =} \OtherTok{FALSE}\NormalTok{)}

\CommentTok{#read back in }
\NormalTok{gbif.POTR <-}\StringTok{ }\KeywordTok{read.csv}\NormalTok{(}\StringTok{"./data/occurrence/GBIF/gbif.POTR.csv"}\NormalTok{, }\DataTypeTok{stringsAsFactors =} \OtherTok{FALSE}\NormalTok{)}

\NormalTok{###convert to shapefile }

\CommentTok{#create object to turn into a shapefile}
\NormalTok{gbif.POTR.shp <-}\StringTok{ }\NormalTok{gbif.POTR }

\CommentTok{#set the fields that represent the coordinates}
\KeywordTok{coordinates}\NormalTok{(gbif.POTR.shp) <-}\StringTok{ }\ErrorTok{~}\StringTok{ }\NormalTok{lon }\OperatorTok{+}\StringTok{ }\NormalTok{lat}

\CommentTok{#convert to spatial points daa frame}
\NormalTok{gbif.POTR.shp <-}\StringTok{ }\KeywordTok{SpatialPointsDataFrame}\NormalTok{(}\DataTypeTok{coords =} \KeywordTok{coordinates}\NormalTok{(gbif.POTR.shp), }
                                       \DataTypeTok{data =} \KeywordTok{data.frame}\NormalTok{(gbif.POTR.shp))}

\CommentTok{# Same as "define projection" in Arc}
\KeywordTok{proj4string}\NormalTok{(gbif.POTR.shp) <-}\StringTok{ }\KeywordTok{CRS}\NormalTok{(}\StringTok{"+proj=longlat +datum=WGS84 +no_defs +ellps=WGS84 +towgs84=0,0,0"}\NormalTok{) }

\CommentTok{#check that CRS is the same as enviro data}
\KeywordTok{identicalCRS}\NormalTok{(gbif.POTR.shp, envStack)}
\end{Highlighting}
\end{Shaded}

\begin{verbatim}
## [1] TRUE
\end{verbatim}

\begin{Shaded}
\begin{Highlighting}[]
\CommentTok{#inspect - we can just use plot() to plot the spatial data}
\KeywordTok{plot}\NormalTok{(gbif.POTR.shp)}
\end{Highlighting}
\end{Shaded}

\includegraphics{SDM_workshop_utah_files/figure-latex/unnamed-chunk-6-1.pdf}
And we have points!

Unfortunately, looks like we have some extraneous points that we didn't
expect here, so we'll need to crop them out.

\begin{Shaded}
\begin{Highlighting}[]
\CommentTok{#make sure we only have points from UT}
\CommentTok{#basically, subsetting the points to the ones that fall  have to do this spatial operation on a spatial object}
\NormalTok{gbif.POTR.shp <-}\StringTok{ }\NormalTok{gbif.POTR.shp[UT.shp, ]}
\KeywordTok{plot}\NormalTok{(gbif.POTR.shp)}
\end{Highlighting}
\end{Shaded}

\includegraphics{SDM_workshop_utah_files/figure-latex/unnamed-chunk-7-1.pdf}

\begin{Shaded}
\begin{Highlighting}[]
\CommentTok{#replace file with the subsetted points in our area of interest}
\CommentTok{#we can access just the data frame portion of the shapefile}
\NormalTok{gbif.POTR <-}\StringTok{ }\NormalTok{gbif.POTR.shp}\OperatorTok{@}\NormalTok{data }\OperatorTok
\StringTok{  }\NormalTok{dplyr}\OperatorTok{::}\KeywordTok{select}\NormalTok{(}\OperatorTok{-}\NormalTok{optional)}

\CommentTok{#export again so we have the most up-to-date version saved! both for our collaborators, and FUTURE US}
\KeywordTok{write.csv}\NormalTok{(gbif.POTR, }\StringTok{"./data/occurrence/GBIF/gbif.POTR.csv"}\NormalTok{, }\DataTypeTok{row.names =} \OtherTok{FALSE}\NormalTok{)}
\end{Highlighting}
\end{Shaded}

Let's plot these on a map!

\begin{Shaded}
\begin{Highlighting}[]
\CommentTok{#to plot points, run this whole chunk }

\NormalTok{maps}\OperatorTok{::}\KeywordTok{map}\NormalTok{(}\DataTypeTok{database =} \StringTok{"state"}\NormalTok{, }\DataTypeTok{regions =} \StringTok{"utah"}\NormalTok{)}
    \KeywordTok{points}\NormalTok{(gbif.POTR.shp)}
\end{Highlighting}
\end{Shaded}

\includegraphics{SDM_workshop_utah_files/figure-latex/unnamed-chunk-8-1.pdf}

\subsubsection{Load occurrence data - from the Forest Service
FIA}\label{load-occurrence-data---from-the-forest-service-fia}

I downloaded and processed these data from the Forest Service website,
which you can access here: \url{https://apps.fs.usda.gov/fia/datamart/}
.

\begin{Shaded}
\begin{Highlighting}[]
\CommentTok{# Normally, you will just load one version of the data to manipulate. But here, I'm showing you a coupel different ways to load in these data.}

\CommentTok{#load shapefile}
\NormalTok{fia.POTR.shp <-}\StringTok{ }\KeywordTok{shapefile}\NormalTok{(}\KeywordTok{here}\NormalTok{(}\StringTok{"data/occurrence/FIA/fia.POTR.shp"}\NormalTok{))}

\CommentTok{#load .csv}
\NormalTok{fia.POTR <-}\StringTok{ }\KeywordTok{read.csv}\NormalTok{(}\KeywordTok{here}\NormalTok{(}\StringTok{"data/occurrence/FIA/FIA_POTR_UT_presAbs.csv"}\NormalTok{))}
\end{Highlighting}
\end{Shaded}

Let's look at these data:

\begin{Shaded}
\begin{Highlighting}[]
\CommentTok{# plot with base R, again needs to be a shapefile to plot like this}
\CommentTok{# color by presence / absence recorded}
\KeywordTok{plot}\NormalTok{(UT.shp)}
    \KeywordTok{points}\NormalTok{(fia.POTR.shp, }\DataTypeTok{pch =} \DecValTok{21}\NormalTok{, }\DataTypeTok{bg =} \StringTok{"white"}\NormalTok{, }\DataTypeTok{cex =} \FloatTok{0.5}\NormalTok{)}
    \KeywordTok{points}\NormalTok{(fia.POTR.shp[fia.POTR}\OperatorTok{$}\NormalTok{presAbs }\OperatorTok{==}\StringTok{ }\DecValTok{1}\NormalTok{,], }\DataTypeTok{pch =} \DecValTok{21}\NormalTok{, }\DataTypeTok{bg =} \StringTok{"dodgerblue"}\NormalTok{)}
\end{Highlighting}
\end{Shaded}

\includegraphics{SDM_workshop_utah_files/figure-latex/unnamed-chunk-10-1.pdf}
The FIA data include points both where POTR was observed, and where it
was not observed.

\subsubsection{Make the two data sources play
together!}\label{make-the-two-data-sources-play-together}

\begin{Shaded}
\begin{Highlighting}[]
\CommentTok{# join the data}

\CommentTok{# Make the fields align!}
\CommentTok{#The GBIF data only represent places where POTR was observed, but we need to add a field that indicates this.}
\CommentTok{#Also, the FIA data do not have a "name" column, so we remove it here}
\NormalTok{gbif.POTR <-}\StringTok{ }\NormalTok{gbif.POTR }\OperatorTok
\StringTok{  }\KeywordTok{mutate}\NormalTok{(}\DataTypeTok{presAbs =} \DecValTok{1}\NormalTok{) }\OperatorTok
\StringTok{  }\KeywordTok{select}\NormalTok{(}\OperatorTok{-}\NormalTok{name)}

\CommentTok{# bind the data frames together }
\NormalTok{POTR.dat <-}\StringTok{ }\KeywordTok{bind_rows}\NormalTok{(fia.POTR, gbif.POTR) }\OperatorTok
\StringTok{  }\KeywordTok{select}\NormalTok{(lon, lat, presAbs) }\CommentTok{# make sure we get only the fields present in both datasets }

\CommentTok{# #### alternately, you can join two shapefiles like so: }
\CommentTok{# # still have to create presAbs field for the gbif data}
\CommentTok{# gbif.POTR.shp@data <- data.frame(gbif.POTR.shp@data[c("lon", "lat")],}
\CommentTok{#                                  presAbs = rep(1, nrow(gbif.POTR.shp)))}
\CommentTok{# }
\CommentTok{# #check that the projections are the same for these data sets}
\CommentTok{# identicalCRS(gbif.POTR.shp, fia.POTR.shp)}
\CommentTok{# }
\CommentTok{# #bind}
\CommentTok{# POTR.dat.shp <- rbind(gbif.POTR.shp, fia.POTR.shp)}

\CommentTok{# plot all together}
\CommentTok{# ggplot can be used to plot }

\CommentTok{# another way to get spatial data - ready for plotting in ggplot. }
\CommentTok{# this is just a list of points that can be rendered as a polygon by ggplot. }
\CommentTok{# I like ggplot, which is part of the "tidyverse", because I think it's a little easier to understand than base R!}
\NormalTok{UT <-}\StringTok{ }\KeywordTok{map_data}\NormalTok{(}\StringTok{"state"}\NormalTok{, }\DataTypeTok{region =} \StringTok{"utah"}\NormalTok{)}

\KeywordTok{ggplot}\NormalTok{() }\OperatorTok{+}
\StringTok{      }\KeywordTok{geom_polygon}\NormalTok{(}\DataTypeTok{data =}\NormalTok{ UT, }\KeywordTok{aes}\NormalTok{(}\DataTypeTok{x=}\NormalTok{long, }\DataTypeTok{y =}\NormalTok{ lat, }\DataTypeTok{group =}\NormalTok{ group), }\DataTypeTok{fill =} \OtherTok{NA}\NormalTok{, }\DataTypeTok{color =} \StringTok{"black"}\NormalTok{) }\OperatorTok{+}\StringTok{ }
\StringTok{      }\KeywordTok{geom_point}\NormalTok{(}\DataTypeTok{data =}\NormalTok{ POTR.dat }\OperatorTok\StringTok{ }\KeywordTok{arrange}\NormalTok{(presAbs), }
                 \KeywordTok{aes}\NormalTok{(}\DataTypeTok{x =}\NormalTok{ lon, }\DataTypeTok{y =}\NormalTok{ lat, }\DataTypeTok{color =} \KeywordTok{factor}\NormalTok{(presAbs))) }\OperatorTok{+}
\StringTok{      }\KeywordTok{coord_fixed}\NormalTok{(}\FloatTok{1.3}\NormalTok{) }\OperatorTok{+}\StringTok{ }
\StringTok{      }\KeywordTok{labs}\NormalTok{(}\DataTypeTok{title =} \StringTok{"FIA and GBIF PRESENCE/ABSENCE DATA - Populus tremuloides"}\NormalTok{)}
\end{Highlighting}
\end{Shaded}

\includegraphics{SDM_workshop_utah_files/figure-latex/unnamed-chunk-11-1.pdf}

\begin{Shaded}
\begin{Highlighting}[]
\CommentTok{#export your data!}
\KeywordTok{write.csv}\NormalTok{(POTR.dat, }\KeywordTok{here}\NormalTok{(}\StringTok{"data/occurrence/full_POTR.csv"}\NormalTok{), }\DataTypeTok{row.names =} \OtherTok{FALSE}\NormalTok{ )}
\end{Highlighting}
\end{Shaded}

\subsubsection{Load forest mask}\label{load-forest-mask}

Another task you might want to do is to only look at one raster, within
the extent of another raster. We won't be using this today, but I have
provided this as an example. We have a layer that represents areas of
forest in Utah, which was downloaded and prepped from :
\url{https://swregap.org/data/landcover/}

\begin{Shaded}
\begin{Highlighting}[]
\CommentTok{#load mask layer}
\NormalTok{forest_mask <-}\StringTok{ }\KeywordTok{raster}\NormalTok{(}\KeywordTok{here}\NormalTok{(}\StringTok{"data/ut_landcover/ut_forestmask.tif"}\NormalTok{))}

\CommentTok{#inspect}
\KeywordTok{plot}\NormalTok{(forest_mask)}
\end{Highlighting}
\end{Shaded}

\includegraphics{SDM_workshop_utah_files/figure-latex/unnamed-chunk-12-1.pdf}

\begin{Shaded}
\begin{Highlighting}[]
\CommentTok{#make sure mask and layer to be masked have same extent and projection (and resolution?) - otherwise, it won't work!}
\KeywordTok{compareRaster}\NormalTok{(forest_mask, envStack_UT)}
\end{Highlighting}
\end{Shaded}

\begin{verbatim}
## [1] TRUE
\end{verbatim}

\begin{Shaded}
\begin{Highlighting}[]
\CommentTok{#perform the operation}
\NormalTok{envStack_mask <-}\StringTok{ }\KeywordTok{mask}\NormalTok{(}\DataTypeTok{x =}\NormalTok{ envStack_UT, }\DataTypeTok{mask =}\NormalTok{ forest_mask, }\DataTypeTok{maskvalue =} \DecValTok{0}\NormalTok{)}

\CommentTok{#inspect}
\KeywordTok{plot}\NormalTok{(envStack_mask)}
\end{Highlighting}
\end{Shaded}

\includegraphics{SDM_workshop_utah_files/figure-latex/unnamed-chunk-12-2.pdf}

So now, if we wanted it-- we only have environmental data for areas
where there is forest in Utah.

\subsubsection{Extract environmental data to
points}\label{extract-environmental-data-to-points}

For species distribution modeling, we need to create a data frame that
has the values for environmental variables at each of our points. We can
do an extract operation to get this information. We do this using the
extract() function in the raster package. You can extract either by
specifying the coordinates, or by using the shapefile

\begin{Shaded}
\begin{Highlighting}[]
\NormalTok{## extract enviro data to points }

\CommentTok{#by specifying coordinates}
\NormalTok{env.dat <-}\StringTok{ }\NormalTok{raster}\OperatorTok{::}\KeywordTok{extract}\NormalTok{(}\DataTypeTok{x =}\NormalTok{ envStack_UT, }\DataTypeTok{y =}\NormalTok{ POTR.dat[,}\KeywordTok{c}\NormalTok{(}\StringTok{"lon"}\NormalTok{, }\StringTok{"lat"}\NormalTok{)])}
\KeywordTok{head}\NormalTok{(env.dat)}
\end{Highlighting}
\end{Shaded}

\begin{verbatim}
##      Annual.Mean.Temperature Mean.Diurnal.Range Isothermality
## [1,]                9.487500           14.79167      37.44726
## [2,]               10.350000           11.41667      31.62512
## [3,]                9.054167           14.00833      34.33415
## [4,]                8.079166           13.24167      36.17942
## [5,]                6.712500           16.45833      43.08464
## [6,]                9.108334           15.66667      34.89235
##      Temperature.Seasonality Max.Temperature.of.Warmest.Month
## [1,]                887.3636                             28.7
## [2,]                904.1747                             28.3
## [3,]                952.7078                             28.7
## [4,]                849.0461                             26.4
## [5,]                795.2562                             25.6
## [6,]               1031.5254                             30.2
##      Min.Temperature.of.Coldest.Month Temperature.Annual.Range
## [1,]                            -10.8                     39.5
## [2,]                             -7.8                     36.1
## [3,]                            -12.1                     40.8
## [4,]                            -10.2                     36.6
## [5,]                            -12.6                     38.2
## [6,]                            -14.7                     44.9
##      Mean.Temperature.of.Wettest.Quarter
## [1,]                            8.233333
## [2,]                           21.183334
## [3,]                           20.133333
## [4,]                           18.383333
## [5,]                           16.283333
## [6,]                           16.616667
##      Mean.Temperature.of.Driest.Quarter
## [1,]                          20.750000
## [2,]                          14.083333
## [3,]                          -1.633333
## [4,]                          11.383333
## [5,]                           9.916667
## [6,]                          -3.950000
##      Mean.Temperature.of.Warmest.Quarter
## [1,]                            20.75000
## [2,]                            21.90000
## [3,]                            20.96667
## [4,]                            18.88333
## [5,]                            16.86667
## [6,]                            21.50000
##      Mean.Temperature.of.Coldest.Quarter Annual.Precipitation
## [1,]                          -1.1666666                  309
## [2,]                          -0.2500001                  287
## [3,]                          -2.6333334                  212
## [4,]                          -1.8500000                  397
## [5,]                          -2.6666667                  316
## [6,]                          -3.9499998                  273
##      Precipitation.of.Wettest.Month Precipitation.of.Driest.Month
## [1,]                             34                            16
## [2,]                             35                            11
## [3,]                             27                            12
## [4,]                             45                            17
## [5,]                             49                            16
## [6,]                             30                            16
##      Precipitation.Seasonality Precipitation.of.Wettest.Quarter
## [1,]                  19.20286                               98
## [2,]                  28.65912                               97
## [3,]                  32.08870                               79
## [4,]                  25.09788                              129
## [5,]                  34.61431                              115
## [6,]                  21.77701                               85
##      Precipitation.of.Driest.Quarter Precipitation.of.Warmest.Quarter
## [1,]                              63                               63
## [2,]                              44                               78
## [3,]                              39                               67
## [4,]                              72                              105
## [5,]                              53                               96
## [6,]                              49                               71
##      Precipitation.of.Coldest.Quarter
## [1,]                               68
## [2,]                               69
## [3,]                               39
## [4,]                               86
## [5,]                               65
## [6,]                               49
\end{verbatim}

\begin{Shaded}
\begin{Highlighting}[]
\KeywordTok{str}\NormalTok{(env.dat)}
\end{Highlighting}
\end{Shaded}

\begin{verbatim}
##  num [1:3861, 1:19] 9.49 10.35 9.05 8.08 6.71 ...
##  - attr(*, "dimnames")=List of 2
##   ..$ : NULL
##   ..$ : chr [1:19] "Annual.Mean.Temperature" "Mean.Diurnal.Range" "Isothermality" "Temperature.Seasonality" ...
\end{verbatim}

\begin{Shaded}
\begin{Highlighting}[]
\CommentTok{# # by using the shapefile which is already spatial }
\CommentTok{# env.dat <- raster::extract(x = envStack_UT, y = POTR.dat.shp)}
\CommentTok{# head(env.dat)}


\KeywordTok{plot}\NormalTok{(env.dat[,}\DecValTok{2}\NormalTok{] }\OperatorTok{~}\StringTok{ }\KeywordTok{factor}\NormalTok{(POTR.dat}\OperatorTok{$}\NormalTok{presAbs))}
\end{Highlighting}
\end{Shaded}

\includegraphics{SDM_workshop_utah_files/figure-latex/unnamed-chunk-13-1.pdf}

\subsection{Prep the model}\label{prep-the-model}

\subsubsection{Prep the data for the
model}\label{prep-the-data-for-the-model}

Unlike the previous steps we've completed, this step is more exclusive
to species distribution modeling. You can run the following chunks of
code, which are specific to the species distribution modeling process

\begin{Shaded}
\begin{Highlighting}[]
\NormalTok{POTR.mod.dat <-}\StringTok{ }\KeywordTok{BIOMOD_FormatingData}\NormalTok{(}\DataTypeTok{resp.var =} \KeywordTok{as.numeric}\NormalTok{(POTR.dat}\OperatorTok{$}\NormalTok{presAbs),}
                                     \DataTypeTok{resp.xy =}\NormalTok{ POTR.dat[, }\KeywordTok{c}\NormalTok{(}\StringTok{"lon"}\NormalTok{, }\StringTok{"lat"}\NormalTok{)],}
                                     \CommentTok{#resp.var = POTR.dat.shp, # for input: can use shapefile with the presence-absence response in the @data slot}
                                     \DataTypeTok{expl.var =} \KeywordTok{stack}\NormalTok{(envStack_UT),}
                                     \CommentTok{#eval.resp.var = ,}
                                     \CommentTok{#PA.strategy = "random", }
                                     \CommentTok{#PA.nb.rep = 0, }
                                     \CommentTok{#PA.nb.absences = 0,}
                                     \DataTypeTok{resp.name =} \StringTok{"Populus.tremuloides"}\NormalTok{)}
\end{Highlighting}
\end{Shaded}

\begin{verbatim}
## 
## -=-=-=-=-=-=-=-=-=-= Populus.tremuloides Data Formating -=-=-=-=-=-=-=-=-=-=
## 
## > No pseudo absences selection !
##       ! No data has been set aside for modeling evaluation
## -=-=-=-=-=-=-=-=-=-=-=-=-=-=-=-=-= Done -=-=-=-=-=-=-=-=-=-=-=-=-=-=-=-=-=
\end{verbatim}

\begin{Shaded}
\begin{Highlighting}[]
\NormalTok{POTR.mod.dat}
\end{Highlighting}
\end{Shaded}

\begin{verbatim}
## 
## -=-=-=-=-=-=-=-=-=-=-=-=-= 'BIOMOD.formated.data' -=-=-=-=-=-=-=-=-=-=-=-=-=
## 
## sp.name =  Populus.tremuloides
## 
##   530 presences,  3331 true absences and  0 undifined points in dataset
## 
## 
##   19 explanatory variables
## 
##  Annual.Mean.Temperature Mean.Diurnal.Range Isothermality  
##  Min.   :-1.133          Min.   : 8.258     Min.   :28.58  
##  1st Qu.: 4.292          1st Qu.:12.950     1st Qu.:36.24  
##  Median : 6.562          Median :14.283     Median :38.01  
##  Mean   : 6.516          Mean   :14.183     Mean   :38.06  
##  3rd Qu.: 8.783          3rd Qu.:15.417     3rd Qu.:39.91  
##  Max.   :15.442          Max.   :19.967     Max.   :45.62  
##  Temperature.Seasonality Max.Temperature.of.Warmest.Month
##  Min.   : 674.5          Min.   :14.50                   
##  1st Qu.: 793.8          1st Qu.:22.00                   
##  Median : 835.2          Median :25.30                   
##  Mean   : 843.5          Mean   :25.13                   
##  3rd Qu.: 889.7          3rd Qu.:28.30                   
##  Max.   :1156.1          Max.   :34.90                   
##  Min.Temperature.of.Coldest.Month Temperature.Annual.Range
##  Min.   :-19.00                   Min.   :27.30           
##  1st Qu.:-13.40                   1st Qu.:34.80           
##  Median :-12.20                   Median :37.60           
##  Mean   :-12.09                   Mean   :37.22           
##  3rd Qu.:-10.80                   3rd Qu.:39.60           
##  Max.   : -3.10                   Max.   :50.70           
##  Mean.Temperature.of.Wettest.Quarter Mean.Temperature.of.Driest.Quarter
##  Min.   :-6.283                      Min.   :-7.550                    
##  1st Qu.: 6.050                      1st Qu.:-1.467                    
##  Median :12.867                      Median : 7.350                    
##  Mean   :11.103                      Mean   : 6.798                    
##  3rd Qu.:16.650                      3rd Qu.:13.917                    
##  Max.   :24.050                      Max.   :23.267                    
##  Mean.Temperature.of.Warmest.Quarter Mean.Temperature.of.Coldest.Quarter
##  Min.   : 8.267                      Min.   :-8.917                     
##  1st Qu.:14.533                      1st Qu.:-5.150                     
##  Median :17.217                      Median :-3.600                     
##  Mean   :17.224                      Mean   :-3.390                     
##  3rd Qu.:19.950                      3rd Qu.:-1.700                     
##  Max.   :26.717                      Max.   : 5.350                     
##  Annual.Precipitation Precipitation.of.Wettest.Month
##  Min.   :155.0        Min.   :21.00                 
##  1st Qu.:312.0        1st Qu.:36.00                 
##  Median :374.0        Median :42.00                 
##  Mean   :372.2        Mean   :42.47                 
##  3rd Qu.:431.0        3rd Qu.:48.00                 
##  Max.   :639.0        Max.   :73.00                 
##  Precipitation.of.Driest.Month Precipitation.Seasonality
##  Min.   : 5.00                 Min.   : 6.004           
##  1st Qu.:14.00                 1st Qu.:14.976           
##  Median :20.00                 Median :20.832           
##  Mean   :20.26                 Mean   :21.254           
##  3rd Qu.:25.00                 3rd Qu.:26.762           
##  Max.   :40.00                 Max.   :40.002           
##  Precipitation.of.Wettest.Quarter Precipitation.of.Driest.Quarter
##  Min.   : 58.0                    Min.   : 23.00                 
##  1st Qu.: 99.0                    1st Qu.: 57.00                 
##  Median :114.0                    Median : 72.00                 
##  Mean   :115.4                    Mean   : 71.33                 
##  3rd Qu.:131.0                    3rd Qu.: 86.00                 
##  Max.   :202.0                    Max.   :124.00                 
##  Precipitation.of.Warmest.Quarter Precipitation.of.Coldest.Quarter
##  Min.   : 41.00                   Min.   : 23.00                  
##  1st Qu.: 79.00                   1st Qu.: 66.00                  
##  Median : 91.00                   Median : 87.00                  
##  Mean   : 94.84                   Mean   : 88.26                  
##  3rd Qu.:107.00                   3rd Qu.:110.00                  
##  Max.   :202.00                   Max.   :177.00                  
## 
## -=-=-=-=-=-=-=-=-=-=-=-=-=-=-=-=-=-=-=-=-=-=-=-=-=-=-=-=-=-=-=-=-=-=-=-=-=-=
\end{verbatim}

\begin{Shaded}
\begin{Highlighting}[]
\KeywordTok{BIOMOD_ModelingOptions}\NormalTok{() }\CommentTok{# need to install java in order to run Maxent.Phillips; we won't be doing this today because it can be pretty finicky!}
\end{Highlighting}
\end{Shaded}

\begin{verbatim}
## 
## -=-=-=-=-=-=-=-=-=-=-=-=  'BIOMOD.Model.Options'  -=-=-=-=-=-=-=-=-=-=-=-=
## 
## 
## GLM = list( type = 'quadratic',
##             interaction.level = 0,
##             myFormula = NULL,
##             test = 'AIC',
##             family = binomial(link = 'logit'),
##             mustart = 0.5,
##             control = glm.control(epsilon = 1e-08, maxit = 50
## , trace = FALSE) ),
## 
## 
## GBM = list( distribution = 'bernoulli',
##             n.trees = 2500,
##             interaction.depth = 7,
##             n.minobsinnode = 5,
##             shrinkage = 0.001,
##             bag.fraction = 0.5,
##             train.fraction = 1,
##             cv.folds = 3,
##             keep.data = FALSE,
##             verbose = FALSE,
##             perf.method = 'cv',
##             n.cores = 1),
## 
## GAM = list( algo = 'GAM_mgcv',
##             type = 's_smoother',
##             k = -1,
##             interaction.level = 0,
##             myFormula = NULL,
##             family = binomial(link = 'logit'),
##             method = 'GCV.Cp',
##             optimizer = c('outer','newton'),
##             select = FALSE,
##             knots = NULL,
##             paraPen = NULL,
##             control = list(nthreads = 1, irls.reg = 0, epsilon = 1e-07
## , maxit = 200, trace = FALSE, mgcv.tol = 1e-07, mgcv.half = 15
## , rank.tol = 1.49011611938477e-08
## , nlm = list(ndigit=7, gradtol=1e-06, stepmax=2, steptol=1e-04, iterlim=200, check.analyticals=0)
## , optim = list(factr=1e+07)
## , newton = list(conv.tol=1e-06, maxNstep=5, maxSstep=2, maxHalf=30, use.svd=0)
## , outerPIsteps = 0, idLinksBases = TRUE, scalePenalty = TRUE
## , efs.lspmax = 15, efs.tol = 0.1, keepData = FALSE, scale.est = fletcher
## , edge.correct = FALSE) ),
## 
## 
## CTA = list( method = 'class',
##             parms = 'default',
##             cost = NULL,
##             control = list(xval = 5, minbucket = 5, minsplit = 5
## , cp = 0.001, maxdepth = 25) ),
## 
## 
## ANN = list( NbCV = 5,
##             size = NULL,
##             decay = NULL,
##             rang = 0.1,
##             maxit = 200),
## 
## SRE = list( quant = 0.025),
## 
## FDA = list( method = 'mars',
##             add_args = NULL),
## 
## MARS = list( type = 'simple',
##              interaction.level = 0,
##              myFormula = NULL,
##              nk = NULL,
##              penalty = 2,
##              thresh = 0.001,
##              nprune = NULL,
##              pmethod = 'backward'),
## 
## RF = list( do.classif = TRUE,
##            ntree = 500,
##            mtry = 'default',
##            nodesize = 5,
##            maxnodes = NULL),
## 
## MAXENT.Phillips = list( path_to_maxent.jar = '/Users/lilaleatherman/Documents/github/sdm_tutorial_utah',
##                memory_allocated = 512,
##                background_data_dir = 'default',
##                maximumbackground = 'default',
##                maximumiterations = 200,
##                visible = FALSE,
##                linear = TRUE,
##                quadratic = TRUE,
##                product = TRUE,
##                threshold = TRUE,
##                hinge = TRUE,
##                lq2lqptthreshold = 80,
##                l2lqthreshold = 10,
##                hingethreshold = 15,
##                beta_threshold = -1,
##                beta_categorical = -1,
##                beta_lqp = -1,
##                beta_hinge = -1,
##                betamultiplier = 1,
##                defaultprevalence = 0.5),
## 
## MAXENT.Tsuruoka = list( l1_regularizer = 0,
##                         l2_regularizer = 0,
##                         use_sgd = FALSE,
##                         set_heldout = 0,
##                         verbose = FALSE)
## -=-=-=-=-=-=-=-=-=-=-=-=-=-=-=-=-=-=-=-=-=-=-=-=-=-=-=-=-=-=-=-=-=-=-=-=-=-=
\end{verbatim}

\begin{Shaded}
\begin{Highlighting}[]
\CommentTok{#myBiomodOptions <- BIOMOD_ModelingOptions(MAXENT.Phillips = list(path_to_maxent.jar = "maxent/maxent.jar"))}
\end{Highlighting}
\end{Shaded}

\subsubsection{Run the model}\label{run-the-model}

\begin{Shaded}
\begin{Highlighting}[]
\NormalTok{POTR.mod <-}\StringTok{ }\KeywordTok{BIOMOD_Modeling}\NormalTok{(}\DataTypeTok{data =}\NormalTok{ POTR.mod.dat, }
                            \CommentTok{#models = c('GLM','GAM','ANN','RF','MAXENT.Tsuruoka'),  }
                            \DataTypeTok{models =} \KeywordTok{c}\NormalTok{(}\StringTok{'GBM'}\NormalTok{,}\StringTok{'ANN'}\NormalTok{,}\StringTok{'RF'}\NormalTok{), }
                            \CommentTok{#SaveObj = TRUE,}
                            \CommentTok{#models.options = myBiomodOptions,}
                            \CommentTok{# , DataSplit = 80}
                            \DataTypeTok{VarImport =} \DecValTok{1}\NormalTok{)}
\end{Highlighting}
\end{Shaded}

\begin{verbatim}
## 
## 
## Loading required library...
## 
## Checking Models arguments...
## 
## Creating suitable Workdir...
## 
##  > No weights : all observations will have the same weight
## 
## 
## -=-=-=-=-=-=-=-=-= Populus.tremuloides Modeling Summary -=-=-=-=-=-=-=-=-=
## 
##  19  environmental variables ( Annual.Mean.Temperature Mean.Diurnal.Range Isothermality Temperature.Seasonality Max.Temperature.of.Warmest.Month Min.Temperature.of.Coldest.Month Temperature.Annual.Range Mean.Temperature.of.Wettest.Quarter Mean.Temperature.of.Driest.Quarter Mean.Temperature.of.Warmest.Quarter Mean.Temperature.of.Coldest.Quarter Annual.Precipitation Precipitation.of.Wettest.Month Precipitation.of.Driest.Month Precipitation.Seasonality Precipitation.of.Wettest.Quarter Precipitation.of.Driest.Quarter Precipitation.of.Warmest.Quarter Precipitation.of.Coldest.Quarter )
## Number of evaluation repetitions : 1
## Models selected : GBM ANN RF 
## 
## Total number of model runs : 3 
## 
## -=-=-=-=-=-=-=-=-=-=-=-=-=-=-=-=-=-=-=-=-=-=-=-=-=-=-=-=-=-=-=-=-=-=-=-=-=-=
## 
## 
## -=-=-=- Run :  Populus.tremuloides_AllData 
## 
## 
## -=-=-=--=-=-=- Populus.tremuloides_AllData_Full 
## 
## Model=Generalised Boosting Regression 
##   2500 maximum different trees and  3  Fold Cross-Validation
\end{verbatim}

\includegraphics{SDM_workshop_utah_files/figure-latex/unnamed-chunk-15-1.pdf}

\begin{verbatim}
## 
##  Evaluating Model stuff...
##  Evaluating Predictor Contributions... 
## 
## Model=Artificial Neural Network 
##   5 Fold Cross Validation + 3 Repetitions
##  Model scaling...
##  Evaluating Model stuff...
##  Evaluating Predictor Contributions... 
## 
## Model=Breiman and Cutler's random forests for classification and regression
##  Evaluating Model stuff...
##  Evaluating Predictor Contributions... 
## 
## -=-=-=-=-=-=-=-=-=-=-=-=-=-=-=-=-= Done -=-=-=-=-=-=-=-=-=-=-=-=-=-=-=-=-=
\end{verbatim}

\begin{Shaded}
\begin{Highlighting}[]
\NormalTok{POTR.mod}
\end{Highlighting}
\end{Shaded}

\begin{verbatim}
## 
## -=-=-=-=-=-=-=-=-=-=-=-=-=-= BIOMOD.models.out -=-=-=-=-=-=-=-=-=-=-=-=-=-=
## 
## Modeling id : 1555529582
## 
## Species modeled : Populus.tremuloides
## 
## Considered variables : Annual.Mean.Temperature Mean.Diurnal.Range 
## Isothermality Temperature.Seasonality Max.Temperature.of.Warmest.Month 
## Min.Temperature.of.Coldest.Month Temperature.Annual.Range 
## Mean.Temperature.of.Wettest.Quarter Mean.Temperature.of.Driest.Quarter 
## Mean.Temperature.of.Warmest.Quarter Mean.Temperature.of.Coldest.Quarter 
## Annual.Precipitation Precipitation.of.Wettest.Month 
## Precipitation.of.Driest.Month Precipitation.Seasonality 
## Precipitation.of.Wettest.Quarter Precipitation.of.Driest.Quarter 
## Precipitation.of.Warmest.Quarter Precipitation.of.Coldest.Quarter
## 
## 
## Computed Models :  Populus.tremuloides_AllData_Full_GBM 
## Populus.tremuloides_AllData_Full_ANN Populus.tremuloides_AllData_Full_RF
## 
## 
## Failed Models :  none
## 
## -=-=-=-=-=-=-=-=-=-=-=-=-=-=-=-=-=-=-=-=-=-=-=-=-=-=-=-=-=-=-=-=-=-=-=-=-=-=
\end{verbatim}

So now we've built a suite of species distribution models! This graph is
one method of evaluating the boosted regression tree model, abbreviated
in the model formulation as ``GBM,'' or ``generalized boosted model.''
We can plot and visualize these in a couple different ways.

Below I use tidyverse techniques to manipulate and plot these
evaluations in a more attractive way, using the ggplot package.

\subsection{View and evaluate the
model}\label{view-and-evaluate-the-model}

\subsubsection{Get model evaluations}\label{get-model-evaluations}

\begin{Shaded}
\begin{Highlighting}[]
\NormalTok{POTR.mod.eval <-}\StringTok{ }\KeywordTok{get_evaluations}\NormalTok{(POTR.mod)}
    \KeywordTok{dimnames}\NormalTok{(POTR.mod.eval)}
\end{Highlighting}
\end{Shaded}

\begin{verbatim}
## [[1]]
## [1] "KAPPA" "TSS"   "ROC"  
## 
## [[2]]
## [1] "Testing.data" "Cutoff"       "Sensitivity"  "Specificity" 
## 
## [[3]]
## [1] "GBM" "ANN" "RF" 
## 
## [[4]]
## [1] "Full"
## 
## [[5]]
## Populus.tremuloides_AllData 
##                   "AllData"
\end{verbatim}

\begin{Shaded}
\begin{Highlighting}[]
\NormalTok{    POTR.mod.eval[}\StringTok{"TSS"}\NormalTok{,}\StringTok{"Testing.data"}\NormalTok{,,,]}
\end{Highlighting}
\end{Shaded}

\begin{verbatim}
##   GBM   ANN    RF 
## 0.637 0.587 0.869
\end{verbatim}

\begin{Shaded}
\begin{Highlighting}[]
\NormalTok{    POTR.mod.eval[}\StringTok{"KAPPA"}\NormalTok{,}\StringTok{"Testing.data"}\NormalTok{,,,]}
\end{Highlighting}
\end{Shaded}

\begin{verbatim}
##   GBM   ANN    RF 
## 0.432 0.338 0.649
\end{verbatim}

\begin{Shaded}
\begin{Highlighting}[]
\NormalTok{    POTR.mod.eval[}\StringTok{"ROC"}\NormalTok{,}\StringTok{"Testing.data"}\NormalTok{,,,]}
\end{Highlighting}
\end{Shaded}

\begin{verbatim}
##   GBM   ANN    RF 
## 0.880 0.814 0.947
\end{verbatim}

\begin{Shaded}
\begin{Highlighting}[]
\CommentTok{# prep and plot these data a little more attractively}
    
\NormalTok{POTR.mod.eval <-}\StringTok{ }
\StringTok{     }\KeywordTok{data.frame}\NormalTok{(POTR.mod.eval[}\DecValTok{1}\OperatorTok{:}\DecValTok{3}\NormalTok{, }\DecValTok{1}\OperatorTok{:}\DecValTok{4}\NormalTok{, ,,]) }\OperatorTok
\StringTok{       }\KeywordTok{rownames_to_column}\NormalTok{(}\StringTok{"metric"}\NormalTok{) }\OperatorTok
\StringTok{       }\KeywordTok{gather}\NormalTok{(}\OperatorTok{-}\NormalTok{metric, }\DataTypeTok{key =} \StringTok{"mod"}\NormalTok{, }\DataTypeTok{value =} \StringTok{"val"}\NormalTok{) }\OperatorTok
\StringTok{       }\KeywordTok{mutate}\NormalTok{(}\DataTypeTok{mod =} \KeywordTok{gsub}\NormalTok{(}\DataTypeTok{pattern =} \StringTok{"Testing.data"}\NormalTok{, }\StringTok{"Testingdata"}\NormalTok{, mod)) }\OperatorTok
\StringTok{       }\KeywordTok{separate}\NormalTok{(mod, }\KeywordTok{c}\NormalTok{(}\StringTok{"val_type"}\NormalTok{, }\StringTok{"model"}\NormalTok{))}

\CommentTok{#inspect evaluation statistics}
\KeywordTok{print}\NormalTok{(}
\NormalTok{POTR.mod.eval }\OperatorTok
\StringTok{  }\KeywordTok{filter}\NormalTok{(val_type }\OperatorTok{==}\StringTok{ "Testingdata"}\NormalTok{) }\OperatorTok
\StringTok{  }\KeywordTok{ggplot}\NormalTok{(}\KeywordTok{aes}\NormalTok{(}\DataTypeTok{x =}\NormalTok{ metric, }\DataTypeTok{y =}\NormalTok{ val, }\DataTypeTok{fill =}\NormalTok{ model)) }\OperatorTok{+}
\StringTok{  }\KeywordTok{geom_bar}\NormalTok{(}\DataTypeTok{stat =} \StringTok{"identity"}\NormalTok{, }\DataTypeTok{position =} \StringTok{"dodge"}\NormalTok{) }\OperatorTok{+}\StringTok{ }
\StringTok{  }\KeywordTok{theme}\NormalTok{(}\DataTypeTok{axis.text.x =} \KeywordTok{element_text}\NormalTok{(}\DataTypeTok{angle =} \DecValTok{90}\NormalTok{, }\DataTypeTok{hjust =} \DecValTok{1}\NormalTok{)) }\OperatorTok{+}\StringTok{ }
\StringTok{  }\KeywordTok{labs}\NormalTok{(}\DataTypeTok{title =} \StringTok{"model evaluation - on testing data"}\NormalTok{,}
       \DataTypeTok{x =} \StringTok{"model type"}\NormalTok{)}
\NormalTok{) }
\end{Highlighting}
\end{Shaded}

\includegraphics{SDM_workshop_utah_files/figure-latex/unnamed-chunk-16-1.pdf}

So, this has reorganized the data so that we are now plotting model
evaluation statistics for each model type, side by side.

We can do the same thing to look at the importance of different
variables in the model. Here, this means how much each environmental
variable contributes to the model.

\subsubsection{Get variable importance}\label{get-variable-importance}

\begin{Shaded}
\begin{Highlighting}[]
\KeywordTok{get_variables_importance}\NormalTok{(POTR.mod)}
\end{Highlighting}
\end{Shaded}

\begin{verbatim}
## , , Full, AllData
## 
##                                       GBM   ANN    RF
## Annual.Mean.Temperature             0.449 0.260 0.126
## Mean.Diurnal.Range                  0.001 0.005 0.011
## Isothermality                       0.003 0.015 0.009
## Temperature.Seasonality             0.004 0.000 0.012
## Max.Temperature.of.Warmest.Month    0.030 0.016 0.064
## Min.Temperature.of.Coldest.Month    0.001 0.010 0.009
## Temperature.Annual.Range            0.001 0.001 0.022
## Mean.Temperature.of.Wettest.Quarter 0.004 0.000 0.016
## Mean.Temperature.of.Driest.Quarter  0.006 0.000 0.016
## Mean.Temperature.of.Warmest.Quarter 0.008 0.134 0.078
## Mean.Temperature.of.Coldest.Quarter 0.016 0.170 0.054
## Annual.Precipitation                0.012 0.079 0.027
## Precipitation.of.Wettest.Month      0.000 0.026 0.006
## Precipitation.of.Driest.Month       0.002 0.150 0.014
## Precipitation.Seasonality           0.005 0.005 0.011
## Precipitation.of.Wettest.Quarter    0.002 0.011 0.012
## Precipitation.of.Driest.Quarter     0.002 0.000 0.023
## Precipitation.of.Warmest.Quarter    0.008 0.005 0.017
## Precipitation.of.Coldest.Quarter    0.009 0.035 0.014
\end{verbatim}

We can also prep and plot this a little more attractively.

\begin{Shaded}
\begin{Highlighting}[]
\CommentTok{# investigate variable importance}
\CommentTok{# still some parsing to do here to visualize}
\NormalTok{var.imp <-}\StringTok{ }\NormalTok{(}\KeywordTok{get_variables_importance}\NormalTok{(POTR.mod))}
\NormalTok{var.imp <-}\StringTok{ }\KeywordTok{data.frame}\NormalTok{(var.imp)}
\CommentTok{# var.imp$var <- names(envStack_aoi)}
\CommentTok{# var.imp}

\NormalTok{var.imp <-}
\StringTok{  }\NormalTok{var.imp }\OperatorTok
\StringTok{  }\KeywordTok{mutate}\NormalTok{(}\DataTypeTok{var =} \KeywordTok{names}\NormalTok{(envStack_UT))}\OperatorTok
\StringTok{  }\KeywordTok{gather}\NormalTok{(}\OperatorTok{-}\NormalTok{var, }\DataTypeTok{key =} \StringTok{"mod"}\NormalTok{, }\DataTypeTok{value =} \StringTok{"val"}\NormalTok{) }\OperatorTok
\StringTok{  }\KeywordTok{separate}\NormalTok{(mod, }\KeywordTok{c}\NormalTok{(}\StringTok{"model_type"}\NormalTok{, }\StringTok{"model_level"}\NormalTok{, }\StringTok{"data_amt"}\NormalTok{)) }

\CommentTok{#plot var importance}
\KeywordTok{print}\NormalTok{(}
\NormalTok{var.imp }\OperatorTok
\StringTok{  }\CommentTok{#mutate(val = ifelse(model_type == "RF", val*10, val)) %>% #multiply RF values by 10 to compare better? idk why these are so low}
\StringTok{  }\KeywordTok{ggplot}\NormalTok{(}\KeywordTok{aes}\NormalTok{(}\DataTypeTok{y =}\NormalTok{ val, }\DataTypeTok{x =} \KeywordTok{reorder}\NormalTok{(var, val), }\DataTypeTok{fill =}\NormalTok{ val, }\DataTypeTok{group =} \KeywordTok{interaction}\NormalTok{(data_amt, model_type))) }\OperatorTok{+}
\StringTok{  }\KeywordTok{geom_bar}\NormalTok{(}\DataTypeTok{stat =} \StringTok{"identity"}\NormalTok{, }\DataTypeTok{position =} \StringTok{"dodge"}\NormalTok{) }\OperatorTok{+}\StringTok{ }
\StringTok{  }\KeywordTok{coord_flip}\NormalTok{() }\OperatorTok{+}
\StringTok{  }\KeywordTok{theme}\NormalTok{(}\DataTypeTok{axis.text.x =} \KeywordTok{element_text}\NormalTok{(}\DataTypeTok{angle =} \DecValTok{90}\NormalTok{, }\DataTypeTok{hjust =} \DecValTok{1}\NormalTok{)) }\OperatorTok{+}\StringTok{ }
\StringTok{  }\KeywordTok{labs}\NormalTok{(}\DataTypeTok{title =} \StringTok{"Variable importance, by model"}\NormalTok{,}
       \DataTypeTok{x =} \StringTok{"variable"}\NormalTok{) }\OperatorTok{+}\StringTok{ }
\StringTok{  }\KeywordTok{facet_grid}\NormalTok{(}\OperatorTok{~}\NormalTok{model_type)}
\NormalTok{)}
\end{Highlighting}
\end{Shaded}

\includegraphics{SDM_workshop_utah_files/figure-latex/unnamed-chunk-18-1.pdf}

\subsubsection{Create an ensemble model-- averaging all of the models
together}\label{create-an-ensemble-model-averaging-all-of-the-models-together}

Lastly in our species distribution modeling process, we create an
ensemble model that averages all the models together. Here, this means
we make a united map or raster of our results. First we'll build the
model, then we'll project it.

\begin{Shaded}
\begin{Highlighting}[]
\NormalTok{myBiomodEM <-}\StringTok{ }\KeywordTok{BIOMOD_EnsembleModeling}\NormalTok{(}
                  \DataTypeTok{modeling.output =}\NormalTok{ POTR.mod,}
                  \DataTypeTok{chosen.models =} \StringTok{'all'}\NormalTok{,}
                  \DataTypeTok{em.by=} \StringTok{'all'}\NormalTok{,}
                  \DataTypeTok{eval.metric =} \KeywordTok{c}\NormalTok{(}\StringTok{'TSS'}\NormalTok{),}
                  \DataTypeTok{eval.metric.quality.threshold =} \KeywordTok{c}\NormalTok{(}\FloatTok{0.6}\NormalTok{),}
                  \DataTypeTok{prob.mean =}\NormalTok{ T,}
                  \DataTypeTok{prob.cv =}\NormalTok{ T,}
                  \DataTypeTok{prob.ci =}\NormalTok{ T,}
                  \DataTypeTok{prob.ci.alpha =} \FloatTok{0.05}\NormalTok{,}
                  \DataTypeTok{prob.median =}\NormalTok{ T,}
                  \DataTypeTok{committee.averaging =}\NormalTok{ T,}
                  \DataTypeTok{prob.mean.weight =}\NormalTok{ T,}
                  \DataTypeTok{prob.mean.weight.decay =} \StringTok{'proportional'}\NormalTok{)}
\end{Highlighting}
\end{Shaded}

\begin{verbatim}
## 
## -=-=-=-=-=-=-=-=-=-=-=-=-= Build Ensemble Models -=-=-=-=-=-=-=-=-=-=-=-=-=
## 
##    ! all models available will be included in ensemble.modeling
##    > Evaluation & Weighting methods summary :
##       TSS over 0.6
## 
## 
##   > mergedAlgo_mergedRun_mergedData ensemble modeling
##    ! Models projections for whole zonation required...
##  > Projecting Populus.tremuloides_AllData_Full_GBM ...
##  > Projecting Populus.tremuloides_AllData_Full_RF ...
## 
##    > Mean of probabilities...
##          Evaluating Model stuff...
##    > Coef of variation of probabilities...
##          Evaluating Model stuff...
##    > Confidence Interval...
##          Evaluating Model stuff...
##          Evaluating Model stuff...
##    > Median of probabilities...
##          Evaluating Model stuff...
##    >  Committee averaging...
##          Evaluating Model stuff...
##    > Probabilities weighting mean...
##        original models scores =  0.637 0.869
##        final models weights =  0.423 0.577
##          Evaluating Model stuff...
## -=-=-=-=-=-=-=-=-=-=-=-=-=-=-=-=-= Done -=-=-=-=-=-=-=-=-=-=-=-=-=-=-=-=-=
\end{verbatim}

\begin{Shaded}
\begin{Highlighting}[]
\NormalTok{myBiomodEM}
\end{Highlighting}
\end{Shaded}

\begin{verbatim}
## 
## -=-=-=-=-=-=-=-=-=-=-= 'BIOMOD.EnsembleModeling.out' -=-=-=-=-=-=-=-=-=-=-=
## 
## sp.name : Populus.tremuloides
## 
## expl.var.names : Annual.Mean.Temperature Mean.Diurnal.Range Isothermality 
## Temperature.Seasonality Max.Temperature.of.Warmest.Month 
## Min.Temperature.of.Coldest.Month Temperature.Annual.Range 
## Mean.Temperature.of.Wettest.Quarter Mean.Temperature.of.Driest.Quarter 
## Mean.Temperature.of.Warmest.Quarter Mean.Temperature.of.Coldest.Quarter 
## Annual.Precipitation Precipitation.of.Wettest.Month 
## Precipitation.of.Driest.Month Precipitation.Seasonality 
## Precipitation.of.Wettest.Quarter Precipitation.of.Driest.Quarter 
## Precipitation.of.Warmest.Quarter Precipitation.of.Coldest.Quarter
## 
## 
## models computed: 
## Populus.tremuloides_EMmeanByTSS_mergedAlgo_mergedRun_mergedData, Populus.tremuloides_EMcvByTSS_mergedAlgo_mergedRun_mergedData, Populus.tremuloides_EMciInfByTSS_mergedAlgo_mergedRun_mergedData, Populus.tremuloides_EMciSupByTSS_mergedAlgo_mergedRun_mergedData, Populus.tremuloides_EMmedianByTSS_mergedAlgo_mergedRun_mergedData, Populus.tremuloides_EMcaByTSS_mergedAlgo_mergedRun_mergedData, Populus.tremuloides_EMwmeanByTSS_mergedAlgo_mergedRun_mergedData
## 
## -=-=-=-=-=-=-=-=-=-=-=-=-=-=-=-=-=-=-=-=-=-=-=-=-=-=-=-=-=-=-=-=-=-=-=-=-=-=
\end{verbatim}

\begin{Shaded}
\begin{Highlighting}[]
\KeywordTok{get_evaluations}\NormalTok{(myBiomodEM)}
\end{Highlighting}
\end{Shaded}

\begin{verbatim}
## $Populus.tremuloides_EMmeanByTSS_mergedAlgo_mergedRun_mergedData
##       Testing.data Cutoff Sensitivity Specificity
## KAPPA        0.603    285      88.491      87.932
## TSS          0.789    246      93.208      85.590
## ROC          0.937    183      98.868      80.036
## 
## $Populus.tremuloides_EMcvByTSS_mergedAlgo_mergedRun_mergedData
##       Testing.data Cutoff Sensitivity Specificity
## KAPPA           NA     NA          NA          NA
## TSS             NA     NA          NA          NA
## ROC             NA     NA          NA          NA
## 
## $Populus.tremuloides_EMciInfByTSS_mergedAlgo_mergedRun_mergedData
##       Testing.data Cutoff Sensitivity Specificity
## KAPPA        0.493  105.0      65.660      89.853
## TSS          0.581   22.0      74.340      83.458
## ROC          0.810   25.5      73.962      84.239
## 
## $Populus.tremuloides_EMciSupByTSS_mergedAlgo_mergedRun_mergedData
##       Testing.data Cutoff Sensitivity Specificity
## KAPPA        0.533    501      86.981      84.839
## TSS          0.795    423      99.623      79.736
## ROC          0.927    424      99.623      79.856
## 
## $Populus.tremuloides_EMmedianByTSS_mergedAlgo_mergedRun_mergedData
##       Testing.data Cutoff Sensitivity Specificity
## KAPPA        0.603    285      88.491      87.932
## TSS          0.789    246      93.208      85.590
## ROC          0.937    183      98.868      80.036
## 
## $Populus.tremuloides_EMcaByTSS_mergedAlgo_mergedRun_mergedData
##       Testing.data Cutoff Sensitivity Specificity
## KAPPA        0.616    747      92.075      87.631
## TSS          0.797    747      92.075      87.631
## ROC          0.927    750      92.075      87.631
## 
## $Populus.tremuloides_EMwmeanByTSS_mergedAlgo_mergedRun_mergedData
##       Testing.data Cutoff Sensitivity Specificity
## KAPPA        0.616  284.0      90.943      87.932
## TSS          0.815  212.0      97.736      83.729
## ROC          0.940  227.5      96.604      85.080
\end{verbatim}

\subsubsection{Project the model
spatially}\label{project-the-model-spatially}

\begin{Shaded}
\begin{Highlighting}[]
\NormalTok{myBiomodProj <-}\StringTok{ }\KeywordTok{BIOMOD_Projection}\NormalTok{(}\DataTypeTok{modeling.output =}\NormalTok{ POTR.mod,}
                    \DataTypeTok{new.env =} \KeywordTok{stack}\NormalTok{(envStack_UT),}
                    \DataTypeTok{proj.name =} \StringTok{'current'}\NormalTok{ ,}
                    \DataTypeTok{selected.models =} \StringTok{'all'}\NormalTok{ , }\CommentTok{# will return separate projections for each model }
                    \DataTypeTok{binary.meth =} \StringTok{'TSS'}\NormalTok{ ,}
                    \DataTypeTok{compress =} \StringTok{'xz'}\NormalTok{ ,}
                    \DataTypeTok{clamping.mask =}\NormalTok{ F,}
                    \DataTypeTok{output.format =} \StringTok{'.grd'}\NormalTok{ )}
\end{Highlighting}
\end{Shaded}

\begin{verbatim}
## 
## -=-=-=-=-=-=-=-=-=-=-=-=-= Do Models Projections -=-=-=-=-=-=-=-=-=-=-=-=-=
## 
##  > Building clamping mask
## 
##  > Projecting Populus.tremuloides_AllData_Full_GBM ...
##  > Projecting Populus.tremuloides_AllData_Full_ANN ...
##  > Projecting Populus.tremuloides_AllData_Full_RF ...
## 
##  > Building TSS binaries
## -=-=-=-=-=-=-=-=-=-=-=-=-=-=-=-=-= Done -=-=-=-=-=-=-=-=-=-=-=-=-=-=-=-=-=
\end{verbatim}

\begin{Shaded}
\begin{Highlighting}[]
\NormalTok{myBiomodProj}
\end{Highlighting}
\end{Shaded}

\begin{verbatim}
## 
## -=-=-=-=-=-=-=-=-=-=-=-= 'BIOMOD.projection.out' -=-=-=-=-=-=-=-=-=-=-=-=
## 
## Projection directory : Populus.tremuloides/current
## 
## 
## sp.name : Populus.tremuloides
## 
## expl.var.names : Annual.Mean.Temperature Mean.Diurnal.Range Isothermality 
## Temperature.Seasonality Max.Temperature.of.Warmest.Month 
## Min.Temperature.of.Coldest.Month Temperature.Annual.Range 
## Mean.Temperature.of.Wettest.Quarter Mean.Temperature.of.Driest.Quarter 
## Mean.Temperature.of.Warmest.Quarter Mean.Temperature.of.Coldest.Quarter 
## Annual.Precipitation Precipitation.of.Wettest.Month 
## Precipitation.of.Driest.Month Precipitation.Seasonality 
## Precipitation.of.Wettest.Quarter Precipitation.of.Driest.Quarter 
## Precipitation.of.Warmest.Quarter Precipitation.of.Coldest.Quarter
## 
## 
## modeling id : 1555529582 ( 
## Populus.tremuloides/Populus.tremuloides.1555529582.models.out )
## 
## models projected : 
## Populus.tremuloides_AllData_Full_GBM, Populus.tremuloides_AllData_Full_ANN, Populus.tremuloides_AllData_Full_RF
## 
## -=-=-=-=-=-=-=-=-=-=-=-=-=-=-=-=-=-=-=-=-=-=-=-=-=-=-=-=-=-=-=-=-=-=-=-=-=-=
\end{verbatim}

\begin{Shaded}
\begin{Highlighting}[]
\KeywordTok{plot}\NormalTok{(myBiomodProj)}
\end{Highlighting}
\end{Shaded}

\includegraphics{SDM_workshop_utah_files/figure-latex/unnamed-chunk-20-1.pdf}

We can also plot these one at a time.

\begin{Shaded}
\begin{Highlighting}[]
\KeywordTok{plot}\NormalTok{(myBiomodProj, }\DataTypeTok{str.grep =} \StringTok{'RF'}\NormalTok{ )}
\end{Highlighting}
\end{Shaded}

\includegraphics{SDM_workshop_utah_files/figure-latex/unnamed-chunk-21-1.pdf}

If we want to plot the map on its own, we can extract the results to a
new object.This is the output that you can save and manipulate.

\begin{Shaded}
\begin{Highlighting}[]
\NormalTok{myCurrentProj <-}\StringTok{ }\KeywordTok{get_predictions}\NormalTok{(myBiomodProj)}

\CommentTok{#combine projections from all models}
\NormalTok{presentResult <-}\StringTok{ }\KeywordTok{calc}\NormalTok{(myCurrentProj,}\DataTypeTok{fun =}\NormalTok{ median); }\CommentTok{#Choose whatever descriptive statistic you'd like}
\KeywordTok{plot}\NormalTok{(presentResult)}
\end{Highlighting}
\end{Shaded}

\includegraphics{SDM_workshop_utah_files/figure-latex/unnamed-chunk-22-1.pdf}

\begin{Shaded}
\begin{Highlighting}[]
\CommentTok{#myCurrentProj}
\KeywordTok{plot}\NormalTok{(myCurrentProj[[}\DecValTok{1}\NormalTok{]], }\DataTypeTok{main =} \StringTok{"Aspen (Populus Tremuloides) in Utah"}\NormalTok{)}
\end{Highlighting}
\end{Shaded}

\includegraphics{SDM_workshop_utah_files/figure-latex/unnamed-chunk-22-2.pdf}

\subsubsection{Save output}\label{save-output}

\begin{Shaded}
\begin{Highlighting}[]
\KeywordTok{dir.create}\NormalTok{(}\KeywordTok{here}\NormalTok{(}\StringTok{"model_output"}\NormalTok{))}
\KeywordTok{dir.create}\NormalTok{(}\KeywordTok{here}\NormalTok{(}\StringTok{"model_output/rasters"}\NormalTok{))}

\CommentTok{# save raster stack}
\KeywordTok{writeRaster}\NormalTok{(myCurrentProj, }\DataTypeTok{filename =} \KeywordTok{here}\NormalTok{(}\StringTok{"model_output/rasters/biomod_out.grd"}\NormalTok{), }\DataTypeTok{options =} \StringTok{"INTERLEAVE=BAND"}\NormalTok{, }\DataTypeTok{overwrite =} \OtherTok{TRUE}\NormalTok{)}

\CommentTok{#load and inspect}
\NormalTok{present_In <-}\StringTok{ }\KeywordTok{stack}\NormalTok{(}\KeywordTok{here}\NormalTok{(}\StringTok{"model_output/rasters/biomod_out.grd"}\NormalTok{))}
\NormalTok{present_In}
\end{Highlighting}
\end{Shaded}

\begin{verbatim}
## class       : RasterStack 
## dimensions  : 601, 601, 361201, 3  (nrow, ncol, ncell, nlayers)
## resolution  : 0.008333333, 0.008333333  (x, y)
## extent      : -114.05, -109.0417, 37, 42.00833  (xmin, xmax, ymin, ymax)
## coord. ref. : +proj=longlat +datum=WGS84 +no_defs +ellps=WGS84 +towgs84=0,0,0 
## names       : Populus.tremuloides_AllData_Full_GBM, Populus.tremuloides_AllData_Full_ANN, Populus.tremuloides_AllData_Full_RF 
## min values  :                                   30,                                   19,                                   0 
## max values  :                                  617,                                  353,                                 938
\end{verbatim}

\subsection{Make a map using ggplot}\label{make-a-map-using-ggplot}

I have included a supplement that will go into more detail, but you can
also make maps using ggplot2 that I think are more straightforward to
customize.

\begin{Shaded}
\begin{Highlighting}[]
\KeywordTok{library}\NormalTok{(rasterVis)}
\KeywordTok{library}\NormalTok{(viridis)}

\CommentTok{#choose one layer}
\NormalTok{RF_model <-}\StringTok{ }\NormalTok{present_In}\OperatorTok{$}\NormalTok{Populus.tremuloides_AllData_Full_RF}

\NormalTok{p =}\StringTok{ }\KeywordTok{gplot}\NormalTok{(RF_model) }\OperatorTok{+}
\StringTok{  }\KeywordTok{geom_tile}\NormalTok{(}\KeywordTok{aes}\NormalTok{(}\DataTypeTok{fill =}\NormalTok{ value)) }\OperatorTok{+}\StringTok{ }
\StringTok{  }\KeywordTok{scale_fill_viridis}\NormalTok{(}\DataTypeTok{na.value =} \StringTok{"white"}\NormalTok{) }\OperatorTok{+}\StringTok{ }
\StringTok{  }\KeywordTok{theme_bw}\NormalTok{() }\OperatorTok{+}\StringTok{ }\KeywordTok{theme}\NormalTok{(}\DataTypeTok{panel.grid.major =} \KeywordTok{element_blank}\NormalTok{(), }\DataTypeTok{panel.grid.minor =} \KeywordTok{element_blank}\NormalTok{()) }\OperatorTok{+}\StringTok{ }
\StringTok{  }\KeywordTok{coord_fixed}\NormalTok{(}\DataTypeTok{ratio =} \FloatTok{1.3}\NormalTok{) }\CommentTok{# sets the xy resolution to a constant value}

\NormalTok{p}
\end{Highlighting}
\end{Shaded}

\includegraphics{SDM_workshop_utah_files/figure-latex/unnamed-chunk-24-1.pdf}


\end{document}
